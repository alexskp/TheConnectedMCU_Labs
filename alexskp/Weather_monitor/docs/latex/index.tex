IoT device for displaying current weather in you city using chip\+Kit Wi-\/\+Fire board. It connects to your Wi\+Fi network and sending G\+ET requests to \href{https://openweathermap.org/}{\tt openweathermap.\+org} weather service to get actual weather in choosed location. There is a guide how to use api of this service\+: \href{https://openweathermap.org/current#format}{\tt openweathermap.\+org/current\#format}.

The chip\+Kit board is setted up to program it with Arduino I\+DE. There is a video guide which explains how to do this\+: \href{https://www.youtube.com/watch?v=DOEdmc57FVU}{\tt youtube.\+com/watch?v=D\+O\+Edmc57\+F\+VU}.

To compile the sketch you need to install a few additional libraries\+: Adafruit\+\_\+\+S\+S\+D1306 and Arduino\+Json. You can do it from library manager in Arduino I\+DE. Adafruit\+\_\+\+S\+S\+D1306 library is used for driving O\+L\+ED display with S\+S\+D1306 driver and Arduino\+Json library is used to parse J\+S\+ON from weather service.

Project was documented using doxygen, so if you have \char`\"{}latex\char`\"{} on your machine you can simply generate project\textquotesingle{}s reference manual in pdf by using \char`\"{}make pdf\char`\"{} command in your console.

